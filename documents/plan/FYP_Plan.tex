\documentclass[]{project_plan}
\bibliographystyle{apalike}
\bibstyle{apalike}
\usepackage{graphicx}
\usepackage{hyperref}
\usepackage{cite}

\newcommand{\bulletPoint}{\hspace{-3.1pt}$\bullet$ \hspace{5pt}}

%---

\def\studentname{Dennis Marjanov}
\def\reportyear{2024}
\def\projecttitle{Human Computer Interaction}
\def\supervisorname{Susnas Sourjah}
\def\degree{BSc (Hons) in Computer Science (Software Engineering)}
\def\fullOrHalfUnit{CS3821 Full Unit}
\def\finalOrInterim{Project Plan}

%---

\begin{document}
\maketitle

%---

\chapter*{Declaration}
This plan has been prepared on the basis of my own work. Where other published
and unpublished source materials have been used, these have been acknowledged.
\vskip3em
Word Count: 2187 words
\vskip3em
Student Name: \studentname
\vskip3em
Date of Submission: 11/10/2024
\vskip3em
Signature: Dennis Marjanov
\vskip0em


\newpage

%---

\tableofcontents\pdfbookmark[0]{Table of Contents}{toc}\newpage

%---

\begin{abstract}
  Human-computer interaction is at the core of how people engage with technology, forming a bridge between human intention and machine execution. This connection determines how effectively individuals can use computers and systems to accomplish tasks, from everyday things to life-critical processes. The design of these interactions must feel natural and intuitive, ensuring that users can easily understand and control complex technological systems. Effective HCI is key to minimizing human error, improving efficiency, and creating seamless experiences for all, from professionals to those less able.

  As human reliance on technology increases, well-designed interfaces become crucial, particularly in high-stakes systems such as hospital equipment and industrial controls. Poorly designed interfaces can lead to confusion, errors, and in some cases, tragic consequences. An example is the Therac-25, a medical linear accelerator designed to treat cancer by delivering targeted radiation (electrons or X-rays) to tumours while minimizing damage to healthy tissue. However, the system relied heavily on software without sufficient hardware safety interlocks, making it vulnerable to software bugs. These issues were compounded by a lack of thorough testing and proper documentation.
  Operators, growing more confident over time, increased the speed at which they input treatment data, including selecting the type of radiation beam. If operators corrected data too quickly after submission, the machine would sometimes deliver incorrect doses of radiation due to a race condition in the software that failed to synchronize inputs properly. This resulted in fatal overdoses, causing the deaths of multiple patients. Although error messages, such as "Malfunction 54,"~\cite{israelski_human_2004} appeared on the system, the error messages were numerical and uninformative which gave the operators no clear guidance on the importance of the error message or implications of ignoring it. This poor design failed to provide feedback leading operators to dismiss critical warnings and lead to tragic deaths.\cite{leveson_investigation_1992}

  Studies have shown that simplification of the interface could be key in these scenarios to reduce cognitive load and present information in a clearer more structured way, this would lower the chances of operator error. Providing more intuitive feedback rather than vague error codes would ensure users better understand system statuses and actions, reducing the risk of mistakes due to confusion.\cite{Alan_Dix_No_Date}
  The Boeing 737 Max crashes were another example of poor HCI, where pilots struggled with unclear feedback from the automated system (MCAS), which led to fatal consequences. The same mitigation strategies apply: simplifying interfaces to reduce cognitive load and providing clear, actionable feedback to help users better understand and manage critical situations.\cite{sgobba_b-737_2019}\cite{spielman_boeing_2021}\cite{wendel_technological_2019}

  Not all mistakes caused by poor Human-Computer Interaction (HCI) are life-threatening, as seen in the confusion caused by the Florida butterfly ballot, but what they all share is the significant role of human psychology. In the case of the butterfly ballot, its poor design caused many voters to accidentally select the wrong candidate, as it failed to align with how users naturally interact with visual information.

  Human psychology plays a significant role in how we process visual information, and research has shown that people tend to follow predictable patterns when scanning layouts. One such pattern is the F-pattern, discovered by the Nielsen Norman Group, where users focus on the upper left area of a screen before scanning downward. Eye-tracking studies support this by showing that users are more likely to engage with content positioned in these focal areas. By understanding and leveraging these natural tendencies, designers can structure interfaces more effectively, guiding users to the correct actions intuitively. This reduces cognitive load, minimizes user frustration, and decreases the likelihood of errors.\cite{Norman_eye_2017}\cite{Nielsen_f-shaped_nodate-1}\cite{Pernice_shaped_nodate}

  Colour theory also plays a key role in influencing user behaviour. Research has shown that colours like red command attention and signal urgency or danger, which is why it's commonly used for stop signs and warnings. Blue, associated with calmness and trust, is often used in corporate and healthcare settings. A real-world application of this is traffic lights, where red indicates danger, green suggests safety, and yellow warns users to slow down. These cues help people instinctively know how to act, reducing cognitive load.\cite{elliot_color_2015}

  Further studies on colour psychology show that colours can also improve accessibility for the less abled. High-contrast colour schemes can help those with cognitive or visual impairments navigate interfaces more easily. For instance, studies found that pink can reduce aggression, leading to its use in prison cells to calm inmates. Similarly, the use of green in hospitals promotes calmness and healing, which has shown positive effects on patient recovery.\cite{team_color_2024}

  To address key HCI challenges, an interactive 3D travel planner will simplify the complex and often overwhelming process of planning trips, where travellers must consider a range of factors such as food preferences, cultural needs, and safety concerns. The main goal of this project is to reduce cognitive load and improve navigation, making the travel planning process as intuitive and simple as possible. By leveraging a 3D globe interface, users can visually explore destinations in a way that mirrors how we naturally engage with the world—through visual cues and spatial understanding. The interface will use colour theory to make decisions easier, with warm colours representing hotter climates and cooler colours for cold regions, while a blue-to-red gradient indicates safety levels. This visual approach not only aligns with human perception, where colours and spatial layout have inherent meaning, but also reduces the mental strain of sorting through complex data.

  The early deliverable will focus on implementing a functional 3D globe that users can navigate, with countries color-coded by climate and safety. For the final deliverable, the planner will include interactive features that allow users to click on countries for detailed travel information and generate assisted itineraries. Additionally, the platform will include a blog section designed according to Nielsen Norman Group's eye-tracking studies, incorporating F-pattern and Z-pattern layouts to enhance user engagement by guiding them through content in a way that matches their natural visual behaviour. By presenting information visually, the planner significantly lowers the complexity of trip planning, making the process intuitive and seamless.

  The second project is an Android-based cooking recipe app designed to meet the needs of elderly individuals and the visually impaired, addressing the challenges they face with reading and following instructions, a task many of us take for granted. The app’s primary goal is to simplify the cooking process by presenting one instruction per page to ensure clarity. Each instruction will be large, high-contrast, and the only content on the screen, preventing any confusion or difficulty in reading. Users will be able to tap or swipe to move between steps, with the app providing immediate tactile feedback to confirm their interaction, making the experience more intuitive. This design reduces cognitive overload by breaking down complex instructions into manageable steps, ensuring a stress-free cooking experience for those who may struggle with traditional recipe formats.

  The early deliverables will focus on creating a user-friendly login screen and a basic recipe selection page, ensuring the interface is simple and easy to navigate. Additionally, it will implement a basic version of the one-instruction-per-page format, allowing users to follow a recipe by moving through instructions step by step. For the final deliverables, the app will include fully developed features, such as tactile feedback and vibrations with each interaction, making the app even more accessible and intuitive. The final version will also offer customizable accessibility settings like adjustable text size and colour contrast to cater to the specific needs of visually impaired users which will be implemented at the start via a setup and tutorial on the app.


  The third and final project is an e-commerce art gallery aimed at creating a platform that prioritizes visual appeal, usability, and a dynamic user experience for both customers and artists. Current art gallery websites often rely on bland, minimalist designs that fail to reflect the personality and style of the artists, particularly those with vibrant or chaotic work. This project balances minimalism with more expressive layouts, tailoring the site’s design to better represent the artwork on display. The platform will feature artist profiles where users can view collections and directly contact artists through a contact form. It will also include a random artwork page, allowing users to explore various pieces of art, with each artwork linking to its respective artist’s profile for further exploration and purchasing options.

  The early deliverables will include a main gallery page featuring a scrollable list of artist profiles and a random artwork page that links to each artist’s collection. The final deliverables will introduce adaptive layouts that adjust based on the artist's style, enhancing the visual experience. The platform will also include a fully functional purchasing system, enabling users to buy artwork directly through the site.



\end{abstract}
\newpage

%---

\chapter{Timeline}

If all goes to plan in the first term, I will focus on researching studies such as color theory and human psychology, specifically examining how users naturally react and interact with devices and web pages. During this time, I will also be learning the necessary technologies and languages required to develop the foundational components of my projects. By the end of this term, the initial designs and interfaces for each project will have been created and user-tested to gather valuable feedback. In the second term, I will concentrate on further developing these projects, incorporating feedback and improvements, followed by final testing and evaluations. This process will culminate in the creation of a comprehensive project report, documenting my development process, results, and HCI-based insights.

\section{Term 1}

\scalebox{1}{
  \begin{tabular}{r |@{\bulletPoint} l}
    Week 1 - 2   & Finalize project plan and research HCI principles for all three projects.         \\
    Week 3       & Design wireframes for the 3D globe, recipe app, and art gallery.                  \\
    Week 4 - 5   & Implement basic 3D globe with navigation and color-coded regions.                 \\
    Week 6 - 7   & Develop the recipe app with a one-step-per-page layout, focusing on accessibility \\
    Week 8       & Test and refine the 3D globe and recipe app based on usability feedback.          \\
    Week 9       & Build the art gallery with dynamic layout for artist profiles.                    \\
    Week 10 - 11 & Conduct early deliverables testing and prepare the report on progress.            \\
  \end{tabular}
}

\section{Term 2}

\scalebox{1}{
  \begin{tabular}{r |@{\bulletPoint} l}
    Week 1 - 2  & Refine 3D globe, adding country details and interactivity, followed by testing.       \\
    Week 3 - 4  & Develop itinerary generation and finalize travel planner.                             \\
    Week 5 - 6  & Complete recipe app, adding tactile feedback and accessibility, conduct final tests   \\
    Week 7 - 8  & Add purchasing features and advanced layouts to the art gallery, followed by testing. \\
    Week 9 - 10 & Finalize all features across the projects, ensuring goals are met via testing.        \\
    Week 11     & Submit final deliverables and project report.                                         \\
  \end{tabular}
}

%---

\chapter{Risks and Mitigations}

In the following section i will discuss Risks that my project faces as well as their mitigations. Risks are an inevitable part of all projects and need to be taken into account throughout the process of the project.

\subsubsection{Loss Of Project Data}
One of the risks in this project is the potential loss of data, such as losing the laptop where all the project code is stored, or experiencing hardware failures. This could result in a major setback or even complete loss of progress. While the likelihood of this happening is moderate, the impact would be high. To mitigate this, I will consistently use Git to back up all project code to a remote repository on GitLab, ensuring the project is always backed up and accessible from any device. By committing regularly and updating the repository, I can reduce the risk of data loss and maintain continuous progress.

\subsubsection{Feature Overload Leading to complexity}
There is a risk of overcomplicating the interface by using too many techniques or features, which could result in a complex, cluttered design that goes against the project’s goal of simplicity. This would increase cognitive load and reduce usability, making the interface difficult for users to navigate. The likelihood of this happening is high, given the temptation to add more features as the project progresses, and the impact would also be high, as it could undermine the core objectives of the project. To prevent this, I will focus on adding only the features that directly enhance the user experience, ensuring simplicity and clarity in the design. By carefully prioritizing and evaluating features based on their value to the user, I can maintain the project's goal of an intuitive, user-friendly interface.

\subsubsection{Time Management and Scope}
The complexity of building multiple interfaces and meeting HCI goals introduces the risk of poor time management, particularly if the project scope expands beyond what is feasible within the available time frame. Overcommitting to additional features could lead to delays or incomplete work. The likelihood of this risk is moderate to high, and its impact would be significant, potentially resulting in incomplete deliverables. To address this, I will break the project into manageable milestones with clear timelines, ensuring there is room for testing and iteration. Regular assessments of progress will help keep the project on track, and any adjustments to the scope will be made early, preventing last-minute complications.

\subsubsection{Technical Skill Gaps}
Certain aspects of the project, such as developing the 3D Globe Travel Planner and Accessibility Overlay, require advanced technical skills that may be beyond my current expertise. This could slow down progress or lead to suboptimal results. The likelihood of encountering this issue is high, and the impact on the project’s timeline and quality could be moderate to high. To address this, I will allocate time early in the project to learn the necessary technologies and research best practices. Additionally, I will seek support from online communities and peers with experience in these areas to ensure that any technical challenges are overcome efficiently.

\subsubsection{Accessibility Feature Complexity}
Implementing comprehensive accessibility features, such as colourblind-friendly palettes or high-contrast modes, could become more complex and time-consuming than anticipated. If these features are not adequately planned, they may cause delays in delivering the final product. To mitigate this, I will plan and design accessibility features early, following established accessibility guidelines like the Web Content Accessibility Guidelines (WCAG) to ensure that the implementation is both efficient and effective.








%---


\bibliography{plan_refs}
\end{document}
\end{article}